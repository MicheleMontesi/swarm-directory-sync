\documentclass{article}

% Initialization
\title{Sincronizzazione dei Dati in un Cluster Docker Swarm con MQTTS}
\author{Michele Montesi \\
        E-Mail: michele.montesi3@studio.unibo.it}

\begin{document}

\maketitle

\section{Introduzione}
In questo documento, verrà presentato il progetto di sincronizzazione 
dei dati in un cluster Docker Swarm utilizzando il protocollo MQTTS. 
Sarà illustrato il contesto applicativo, il problema specifico da risolvere e 
la soluzione implementata, comprese le tecnologie e le configurazioni adottate.

\section{Contesto Applicativo}
Il progetto parte da un ambiente Docker Swarm, che è un sistema di orchestrazione 
di container che permette di gestire un cluster di nodi Docker. Inoltre, si assume 
che sia stato configurato un servizio di messaggistica MQTT (Message Queuing Telemetry Transport) 
su tutti i nodi del cluster, noto come MQTTS (MQTT over TLS/SSL). 
L'utilizzo di MQTTS fornisce una comunicazione sicura e cifrata tra i nodi del cluster Swarm.

L'obiettivo del progetto è quello di ottenere un cluster Docker Swarm in cui il 
servizio MQTTS sia in esecuzione su tutti i nodi del cluster. Inoltre, è necessario 
garantire la persistenza dei dati e la loro sincronizzazione tra i nodi, utilizzando 
l'utility lsyncd.

\section{Problema Specifico}
Il problema che si presenta è la necessità di sincronizzare i dati tra i nodi del 
cluster Docker Swarm. A causa della natura distribuita del cluster, con i suoi nodi 
indipendenti, potrebbero verificarsi situazioni in cui i dati siano inconsistenti o disallineati 
tra i vari nodi. È fondamentale risolvere questo problema per garantire l'integrità e la coerenza 
dei dati nell'ambiente Docker Swarm.

\section{Soluzione Implementata}
Per affrontare il problema della sincronizzazione dei dati tra i nodi del 
cluster Docker Swarm, è stata adottata la soluzione seguente:

\subsection{MQTTS}
Il servizio di messaggistica MQTT è stato configurato su tutti i nodi del cluster 
Swarm utilizzando il protocollo MQTTS, che fornisce una connessione sicura e cifrata. 
L'utilizzo di MQTTS garantisce la riservatezza e l'autenticazione durante la 
comunicazione tra i nodi.

\subsection{lsyncd}
Per garantire la sincronizzazione dei dati tra i nodi del cluster, 
è stata utilizzata l'utility lsyncd (Live Syncing Daemon). Lsyncd è uno strumento 
che permette di monitorare le modifiche apportate ai file in una directory e di 
propagarle automaticamente a una o più destinazioni remote.

È stata configurata un'apposita regola di sincronizzazione in lsyncd per monitorare 
le directory dei dati all'interno di ciascun nodo del cluster Docker Swarm e propagare 
automaticamente le modifiche ai nodi rimanenti. In questo modo, qualsiasi modifica 
effettuata su un nodo verrà sincronizzata in modo automatico e trasparente a tutti 
gli altri nodi del cluster.

\subsection{Persistenza dei Dati}
Per garantire la persistenza dei dati nel cluster Docker Swarm, è stato 
configurato un volume Docker per le directory dei dati. I volumi Docker 
consentono di separare i dati dal container, consentendo loro di persistere 
anche in caso di riavvio del container o del nodo. Ciò garantisce che i dati 
rimangano disponibili anche in situazioni di fallimento o manutenzione.

\section{Conclusioni}
In questa relazione è stato presentato il progetto di sincronizzazione 
dei dati in un cluster Docker Swarm con MQTTS. Il progetto è stato motivato 
dal problema della sincronizzazione dei dati tra i nodi indipendenti del cluster Swarm. 
È stata presentata la soluzione implementata, che utilizza il protocollo MQTTS 
per la comunicazione sicura e l'utility lsyncd per la sincronizzazione automatica 
dei dati tra i nodi. Si è inoltre sottolineata l'importanza della sincronizzazione 
dei dati in un ambiente distribuito come Docker Swarm per garantire l'integrità e 
la coerenza dei dati tra i nodi del cluster.

\end{document}
