\documentclass[a4paper, 12pt]{article}
\usepackage{listings} % Per l'inserimento del codice
\usepackage{xcolor} % Per la colorazione del codic

\lstset{
  language=bash, % Linguaggio del codice
  basicstyle=\fontsize{11}{12}\ttfamily, % Stile del testo del codice
  showstringspaces=false % Non mostrare i caratteri di spazio come caratteri speciali
}

% Initialization
\title{Sincronizzazione dei Dati in un Cluster Docker Swarm con MQTTS}
\author{Michele Montesi \\
        E-Mail: michele.montesi3@studio.unibo.it}

% Environments
\newenvironment{packed_itemize}{
\begin{itemize}
        \setlength{\itemsep}{1pt}
        \setlength{\parskip}{0pt}
        \setlength{\parsep}{0pt}
}{\end{itemize}}

\begin{document}

\maketitle

\section{Introduzione}
In questo documento, verrà presentato il progetto di sincronizzazione 
dei dati in un cluster Docker Swarm che utilizza il protocollo MQTTS. 
Sarà illustrato il contesto applicativo, il problema specifico da risolvere e 
la soluzione implementata, comprese le tecnologie e le configurazioni adottate.

\section{Contesto Applicativo}
Il progetto parte da un ambiente Docker Swarm, che è un sistema di orchestrazione 
di container che permette di gestire un cluster di nodi Docker. Inoltre, si assume 
che sia stato configurato un servizio di messaggistica MQTT (Message Queuing Telemetry Transport) 
su tutti i nodi del cluster, noto come MQTTS (MQTT over TLS/SSL). 
L'utilizzo di MQTTS fornisce una comunicazione sicura e cifrata tra i nodi del cluster Swarm.

L'obiettivo del progetto è quello di ottenere un cluster Docker Swarm in cui il 
servizio MQTTS sia in esecuzione su tutti i nodi del cluster, utilizzando gli stessi dati. 
Inoltre, è necessario garantire la persistenza dei dati e la loro sincronizzazione tra i nodi, 
utilizzando l'utility lsyncd.

\newpage
\section{Problema Specifico}
Il problema che si presenta è la necessità di sincronizzare i dati tra i nodi del 
cluster Docker Swarm. A causa della natura distribuita del cluster, con i suoi nodi 
indipendenti, potrebbero verificarsi situazioni in cui i dati siano inconsistenti o disallineati 
tra i vari nodi. È fondamentale risolvere questo problema per garantire l'integrità e la coerenza 
dei dati nell'ambiente Docker Swarm.
In questo caso specifico non è importante la sincronizzazione dei dati in real time, in quanto
i questi sono utilizzati per la configurazione di un servizio di messaggistica,
quindi sono statici e questo limite non è un problema.

\section{Soluzione Implementata}
Per affrontare il problema della sincronizzazione dei dati tra i nodi del 
cluster Docker Swarm, è stata adottata la soluzione seguente:

\subsection{lsyncd}
Per garantire la sincronizzazione dei dati tra i nodi del cluster, 
è stata utilizzata l'utility lsyncd (Live Syncing Daemon). lsyncd è uno strumento 
che permette di monitorare le modifiche apportate ai file in una directory e di 
propagarle automaticamente a una o più destinazioni remote.

È stata configurata un'apposita regola di sincronizzazione in lsyncd per monitorare 
le directory dei dati all'interno di ciascun nodo del cluster Docker Swarm e propagare 
automaticamente le modifiche ai nodi rimanenti. In questo modo, qualsiasi modifica 
effettuata su un nodo verrà sincronizzata in modo automatico e trasparente a tutti 
gli altri nodi del cluster.

\subsection{Ansible}
Per automatizzare l'installazione di lsyncd su tutti i nodi del cluster Docker Swarm, 
è stato utilizzato Ansible. Ansible è un framework di automazione IT che 
semplifica la distribuzione, la configurazione e la gestione delle applicazioni e delle 
infrastrutture. Attraverso l'utilizzo di playbook Ansible, è stato possibile definire e 
eseguire in modo dichiarativo le operazioni di installazione e configurazione di lsyncd 
su tutti i nodi del cluster.

\newpage
\section{Sviluppo degli Script Ansible}
Per lo sviluppo degli script Ansible utilizzati in questo progetto, 
sono stati seguiti i seguenti passaggi:

\subsection{Struttura del Progetto}
Il progetto Ansible è stato organizzato seguendo una struttura standard, 
con i seguenti elementi principali:
\begin{packed_itemize}
  \item \texttt{inventory.yml}: Un file di inventario che elenca tutti i nodi del 
    cluster Docker Swarm su cui verrà eseguita l'installazione di lsyncd.
  \item \texttt{lsyncdInstall.yml}: Il playbook che gestisce l'installazione
    del pacchetto lsyncd su tutti i nodi del cluster Docker Swarm.
  \item \texttt{lsyncdConf.yml}: Il playbook principale che definisce le operazioni da 
    eseguire per la configuraizone di lsyncd.
  \item \texttt{roles/}: Una directory che contiene i ruoli Ansible. 
    Ogni ruolo rappresenta una specifica funzionalità o configurazione da applicare 
    sui nodi del cluster.
  \item \texttt{roles/setup-lsyncd/}: Il ruolo che definisce la configurazione
    di lsyncd su ogni nodo del cluster Docker Swarm.
  \item \texttt{roles/start-lsyncd/}: Il ruolo che definisce l'avvio del
    servizio lsyncd su ogni nodo del cluster Docker Swarm.
\end{packed_itemize}

\subsection{Definizione dei Ruoli}
Per definire i ruoli Ansible per la configurazione e l'avvio di lsyncd, 
sono state eseguite le seguenti attività:
\begin{packed_itemize}
  \item Creazione delle directory:
  \begin{packed_itemize}
    \item \texttt{roles/setup-lsyncd/}
    \item \texttt{roles/start-lsyncd/}
  \end{packed_itemize}
  \item Definizione di un file \texttt{tasks/main.yml} all'interno del ruoli. 
    Questo file contiene l'elenco delle attività da eseguire per installare e configurare lsyncd.
  \item Definizione di file di configurazione di lsyncd all'interno del ruolo \\ 
    \texttt{setup-lsyncd}: \texttt{templates/lsyncd.conf.j2}.
  \item Definizione di file di variabili all'interno di \texttt{setup-lsyncd} per definire
    i percorsi da sincronizzare.
\end{packed_itemize}

\subsection{Utilizzo dei Playbook}
Per eseguire l'installazione, e poi la configurazione, di lsyncd su tutti i nodi del 
cluster Docker Swarm, è sufficiente eseguire i seguenti comandi:
\begin{lstlisting}
ansible-playbook -i inventory.ini -vv lsyncdInstall.yml
ansible-playbook -i inventory.ini -vv lsyncdConf.yml
\end{lstlisting}

\section{Persistenza dei Dati}
Per garantire la persistenza dei dati nel cluster Docker Swarm, è stato 
configurato un volume Docker per le directory dei dati. I volumi Docker 
consentono di separare i dati dal container, consentendo loro di persistere 
anche in caso di riavvio del container o del nodo. Ciò garantisce che i dati 
rimangano disponibili anche in situazioni di fallimento o manutenzione.
Questi dati saranno sincronizzati attraverso lo script ansible descritto in precedenza.

\section{Conclusioni}
In questa relazione è stato presentato il progetto di sincronizzazione 
dei dati in un cluster Docker Swarm con MQTTS tramite lsyncd. Il progetto è stato motivato 
dal problema della sincronizzazione dei dati tra i nodi indipendenti del cluster Swarm. 
È stata presentata la soluzione implementata, l'utility lsyncd per la sincronizzazione automatica 
dei dati tra i nodi. Si è inoltre sottolineata l'importanza della sincronizzazione 
dei dati in un ambiente distribuito come Docker Swarm per garantire l'integrità e 
la coerenza dei dati tra i nodi del cluster.

\end{document}
