\documentclass[a4paper, 12pt]{article}
\usepackage{listings} % For code insertion
\usepackage{xcolor} % For code highlighting

\lstset{
  language=bash, % Code language
  basicstyle=\fontsize{11}{12}\ttfamily, % Code text style
  showstringspaces=false % Don't show spaces as special characters
}

% Initialization
\title{Data Synchronization in a Docker Swarm Cluster with MQTTS}
\author{Michele Montesi \\
        E-Mail: michele.montesi3@studio.unibo.it}

% Environments
\newenvironment{packed_itemize}{
\begin{itemize}
        \setlength{\itemsep}{1pt}
        \setlength{\parskip}{0pt}
        \setlength{\parsep}{0pt}
}{\end{itemize}}

\begin{document}

\maketitle

\section{Introduction}
This document presents the project of data synchronization in a Docker Swarm cluster that uses 
the MQTTS protocol. It describes the application context, the specific problem 
to solve, and the implemented solution, including the technologies and configurations adopted.

\section{Application Context}
The project starts from a Docker Swarm environment, which is a container orchestration 
system that allows managing a cluster of Docker nodes. Additionally, it is assumed that an 
MQTT (Message Queuing Telemetry Transport) messaging service has been configured on all 
cluster nodes, known as MQTTS (MQTT over TLS/SSL). The use of MQTTS provides secure and 
encrypted communication between the Swarm cluster nodes.

The project's goal is to achieve a Docker Swarm cluster where the MQTTS service is 
running on all cluster nodes using the same data. Furthermore, it is necessary to ensure data 
persistence and synchronization between nodes using the lsyncd utility.

\newpage
\section{Specific Problem}
The problem at hand is the need to synchronize data between Docker Swarm cluster nodes. 
Due to the distributed nature of the cluster with its independent nodes, situations may 
arise where data is inconsistent or out-of-sync across the nodes. Resolving this problem 
is essential to ensure the integrity and consistency of data in the Docker Swarm environment.
In this specific case it is not important to synchronize data in real time, 
as these are used to configure a messaging service. So they are static and this limit 
is not a problem.

\section{Implemented Solution}
To address the data synchronization problem between Docker Swarm cluster nodes, the 
following solution has been adopted:

\subsection{lsyncd}
To ensure data synchronization between cluster nodes, the lsyncd (Live Syncing Daemon) 
utility has been used. lsyncd is a tool that monitors file changes in a directory and 
automatically propagates them to one or more remote destinations.

A dedicated synchronization rule has been configured in lsyncd to monitor the data 
directories within each Docker Swarm node and automatically propagate changes to the 
remaining nodes. This way, any modification made on one node will be automatically 
and transparently synchronized across all other cluster nodes.

\subsection{Ansible}
To automate the installation of lsyncd on all Docker Swarm cluster nodes, 
Ansible has been used. Ansible is an IT automation framework that simplifies 
application and infrastructure deployment, configuration, and management. 
By utilizing Ansible playbooks, it was possible to declaratively define and execute 
the installation and configuration operations of lsyncd on all cluster nodes.

\newpage
\section{Development of Ansible Scripts}
The following steps were followed for the development of Ansible scripts used in this project:

\subsection{Project Structure}
The Ansible project was organized following a standard structure, consisting of the following main
elements:
\begin{packed_itemize}
  \item \texttt{inventory.yml}: An inventory file listing all Docker Swarm 
    cluster nodes where lsyncd installation will be performed.
  \item \texttt{lsyncdInstall.yml}: The playbook that handles the installation 
    of the lsyncd package on all Docker Swarm cluster nodes.
  \item \texttt{lsyncdConf.yml}: The main playbook that defines the operations 
    to be executed for lsyncd configuration.
  \item \texttt{roles/}: A directory containing Ansible roles. 
    Each role represents a specific functionality or configuration to be applied on cluster nodes.
  \item \texttt{roles/setup-lsyncd/}: The role that defines lsyncd configuration 
    on each Docker Swarm cluster node.
  \item \texttt{roles/start-lsyncd/}: The role that defines the start of the lsyncd 
    service on each Docker Swarm cluster node.
\end{packed_itemize}

\subsection{Defining Roles}
To define Ansible roles for lsyncd configuration and startup, the following 
activities were performed:
\begin{packed_itemize}
  \item Creation of directories:
  \begin{packed_itemize}
    \item \texttt{roles/setup-lsyncd/}
    \item \texttt{roles/start-lsyncd/}
  \end{packed_itemize}
  \item Definition of a \texttt{tasks/main.yml} file within each role. This file 
    contains a list of tasks to install and configure lsyncd.
  \item Definition of lsyncd configuration files within the \texttt{setup-lsyncd} 
    role: \texttt{templates/lsyncd.conf.j2}.
  \item Definition of variable files within \texttt{setup-lsyncd} to define the paths 
    to be synchronized.
\end{packed_itemize}

\subsection{Using Playbooks}
To perform the installation and configuration of lsyncd on all Docker Swarm cluster nodes, 
you can execute the following commands:
\begin{lstlisting}
ansible-playbook -i inventory.ini -vv lsyncdInstall.yml
ansible-playbook -i inventory.ini -vv lsyncdConf.yml
\end{lstlisting}

\section{Data Persistence}
To ensure data persistence in the Docker Swarm cluster, a Docker volume has been 
configured for data directories. Docker volumes allow separating data from the container, 
enabling them to persist even in the event of container or node restart. This guarantees 
that the data remains available even in failure or maintenance situations. This data will 
be synchronized using the Ansible script described earlier.

\section{Conclusion}
This report presented the project of data synchronization in a Docker Swarm cluster 
with MQTTS using lsyncd. The project was motivated by the data synchronization problem 
between independent nodes in the Swarm cluster. The implemented solution, the 
lsyncd utility for automatic data synchronization between nodes, was introduced. 
The importance of data synchronization in a distributed environment like Docker 
Swarm was emphasized to ensure data integrity and consistency across cluster nodes.

\end{document}