\documentclass{article}

% Initialization
\title{Synchronizing Data in a Docker Swarm Cluster with MQTTS}
\author{Michele Montesi \\
        E-Mail: michele.montesi3@studio.unibo.it}

\begin{document}

\maketitle

\section{Introduction}
This document presents the project of synchronizing data in a Docker Swarm 
cluster using the MQTTS protocol. It will illustrate the application context, 
the specific problem to be solved, and the implemented solution, including the 
technologies and configurations adopted.

\section{Application Context}
The project starts from a Docker Swarm environment, which is a container 
orchestration system that manages a cluster of Docker nodes. It is also assumed 
that an MQTT (Message Queuing Telemetry Transport) messaging service has been configured 
on all nodes of the cluster, known as MQTTS (MQTT over TLS/SSL). The use of MQTTS provides 
secure and encrypted communication between the nodes of the Swarm cluster.

The goal of the project is to achieve a Docker Swarm cluster where the MQTTS service 
is running on all nodes of the cluster. Furthermore, it is necessary to ensure data 
persistence and synchronization between nodes using the lsyncd utility.

\section{Specific Problem}
The specific problem is the need to synchronize data between nodes in the Docker 
Swarm cluster. Due to the distributed nature of the cluster, with its independent nodes, 
situations may arise where data is inconsistent or out of sync across the various nodes. 
It is essential to solve this problem to ensure data integrity and consistency in the 
Docker Swarm environment.

\section{Implemented Solution}
To address the problem of data synchronization between nodes in the Docker Swarm cluster, 
the following solution has been adopted:

\subsection{MQTTS}
The MQTT messaging service has been configured on all nodes of the Swarm cluster using 
the MQTTS protocol, which provides secure and encrypted connections. The use of MQTTS 
ensures confidentiality and authentication during communication between the nodes.

\subsection{lsyncd}
To ensure data synchronization between nodes in the cluster, the lsyncd (Live Syncing Daemon) 
utility has been utilized. Lsyncd is a tool that monitors file changes in a directory and 
automatically propagates them to one or more remote destinations.

A specific synchronization rule has been configured in lsyncd to monitor the data 
directories within each Docker Swarm node and automatically propagate any changes to 
the remaining nodes. This way, any modifications made on one node will be automatically 
and transparently synchronized with all other nodes in the cluster.

\subsection{Data Persistence}
To ensure data persistence in the Docker Swarm cluster, a Docker volume has been 
configured for the data directories. Docker volumes allow data to be separated from the 
container, enabling persistence even in the event of container or node restarts. 
This ensures that data remains available even in failure or maintenance scenarios.

\section{Conclusion}
This report presented the project of synchronizing data in a Docker Swarm cluster with MQTTS. 
The project was motivated by the problem of synchronizing data between independent nodes in 
the Swarm cluster. The implemented solution, which utilizes the MQTTS protocol for secure 
communication and the lsyncd utility for automatic data synchronization between nodes, was 
described. The importance of data synchronization in a distributed environment like Docker 
Swarm was emphasized to ensure data integrity and consistency across the cluster's nodes.

\end{document}
